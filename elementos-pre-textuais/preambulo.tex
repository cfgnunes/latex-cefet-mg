% ------------------------------------------------------------------------------
% ATENÇÃO:
% Caso algum campo não se aplique ao seu documento - por exemplo, em seu
% trabalho não houve coorientador - não comente o campo, apenas deixe vazio,
% assim: \campo{}
% ------------------------------------------------------------------------------

% ------------------------------------------------------------------------------
% Dados do trabalho acadêmico
% ------------------------------------------------------------------------------

\titulo{Título do Trabalho}
%\title{Title in English}
\subtitulo{Subtítulo do trabalho}
\autor{Nome completo do autor}
\local{Belo Horizonte}
\data{Maio de 2020} % Normalmente se usa apenas mês e ano

% ------------------------------------------------------------------------------
% Natureza do trabalho acadêmico
% Use apenas uma das opções: Tese (p/ Doutorado), Dissertação (p/ Mestrado) ou
% Projeto de Qualificação (p/ Mestrado ou Doutorado), Trabalho de Conclusão de
% Curso (Graduação)
% ------------------------------------------------------------------------------

\projeto{Projeto de Qualificação}

% ------------------------------------------------------------------------------
% Título acadêmico
% Use apenas uma das opções:
% - Se a natureza for Tese, coloque Doutor
% - Se a natureza for Dissertação, coloque Mestre
% - Se a natureza for Projeto de Qualificação, coloque Mestre ou Doutor
% - Se a natureza for Trabalho de Conclusão de Curso, coloque Bacharel
% ------------------------------------------------------------------------------

\tituloAcademico{Doutor}

% ------------------------------------------------------------------------------
% Área de concentração e linha de pesquisa
% Observação: Indique o nome da área de concentração e da linha de pesquisa do
% programa de Pós-graduação nas quais este trabalho se insere. Se a natureza
% for Trabalho de Conclusão de Curso, deixe ambos os campos vazios.
% ------------------------------------------------------------------------------

\areaconcentracao{Modelagem Matemática e Computacional.}
\linhapesquisa{Métodos Matemáticos Aplicados.}

% ------------------------------------------------------------------------------
% Dados da instituição
% Observação: A logomarca da instituição deve ser colocada na mesma pasta que
% foi colocada o documento principal com o nome de "logoInstituicao".
% O formato pode ser: pdf, eps, jpg ou png. Se a natureza for Trabalho de
% Conclusão de Curso, coloque em "programa' o nome do curso de graduação.
% ------------------------------------------------------------------------------

\instituicao{Centro Federal de Educação Tecnológica de Minas Gerais}
\programa{Nome do programa ou curso}
\logoinstituicao{0.2}{figuras/logo-instituicao.pdf}

% ------------------------------------------------------------------------------
% Dados do(s) orientador(es)
% ------------------------------------------------------------------------------

\orientador{Nome do orientador}
%\orientador[Orientadora:]{Nome da orientadora}
\instOrientador{Instituição do orientador}

\coorientador{Nome do coorientador}
%\coorientador[Coorientadora:]{Nome da coorientadora}
\instCoorientador{Instituição do coorientador}

% ------------------------------------------------------------------------------
% Folha de Rosto
% ------------------------------------------------------------------------------

% Trabalho de Conclusão de Curso
%\preambulo{{\imprimirprojeto} apresentado ao Curso de Engenharia de Computação do Centro Federal de Educação Tecnológica de Minas Gerais, como requisito parcial para a obtenção do título de {\imprimirtituloAcademico} em Engenharia de Computação.}

% Projeto de qualificação de Mestrado ou Doutorado
\preambulo{{\imprimirprojeto} apresentado ao Programa de Pós-graduação em Modelagem Matemática e Computacional do Centro Federal de Educação Tecnológica de Minas Gerais, como requisito parcial para a obtenção do título de {\imprimirtituloAcademico} em Modelagem Matemática e Computacional.}

% Dissertação de Mestrado
%\preambulo{{\imprimirprojeto} apresentada ao Programa de Pós-graduação em Modelagem Matemática e Computacional do Centro Federal de Educação Tecnológica de Minas Gerais, como requisito parcial para a obtenção do título de {\imprimirtituloAcademico} em Modelagem Matemática e Computacional.}

% Tese de Doutorado
%\preambulo{{\imprimirprojeto} apresentada ao Programa de Pós-graduação em Modelagem Matemática e Computacional do Centro Federal de Educação Tecnológica de Minas Gerais, como requisito parcial para a obtenção do título de {\imprimirtituloAcademico} em Modelagem Matemática e Computacional.}

% ------------------------------------------------------------------------------
% Folha de aprovação
% ------------------------------------------------------------------------------

\textopadraofolhadeaprovacao{Esta folha deverá ser substituída pela cópia digitalizada da folha de aprovação fornecida.}
